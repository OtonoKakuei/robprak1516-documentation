\chapter{Introduction}
\section{Motivation}

Imagine you were sitting in your office, dozing off, ready to sleep. As soon as you fell asleep, someone pulled a prank on you and took you to a different room. Shortly afterwards you woke up confused not knowing where you are. What would you do in this situation? You would probably look around to check if you are still inside the building. Then you would try to pinpoint your location based on what you see inside the room. If that was not enough information you could go outside and start exploring the environment. If you have been here before, you would probably be able to recognize some similarities and find out where you are. Then you could just go back to your office, and continue where you left off. It might seem to be an intuitive task for you, however robots cannot solve it quite as easily.

\section{Objectives}

In this work we would like to tackle the kidnapping problem on robots. The first step is to detect whether a kidnapping situation occurred and the second step is to relocalize itself. Detection can be done by determining whether some information from the sensors is missing or whether the perceived environment does not match with the expected one. Recovery can be done by exploring the surroundings while gathering  information of it. If any known features are found, they will be matched with the robot's database. Successful matching means that the robot can use the results to estimate its position in the environment. Otherwise, the exploration continues.

\chapter{Fundamentals}
\section{ROS}
\section{MCA}
\section{Kinect}
\section{Laser Scanner}
\section{ASAP (Advanced Shared Autonomy Platform)}
