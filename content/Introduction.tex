\chapter{Introduction}
\section{Motivation}

\section{Objectives}

\chapter{Fundamentals}
\section{ROS}
\section{MCA}
\section{Kinect}
\section{Laser Scanner}
\section{ASAP (Advanced Shared Autonomy Platform)}

\section{Kalman Filter}
Many problems in the domain of artificial intelligence require the prediction of future states by evaluating measurements. Around the 1960s Peter Swerling, Rudolf E. Kálmán, Thorvald N. Thiele and Richard Bucy described an algorithm for this purpose. It is now known as the \textit{Kalman filter}.

In the field of robotics Kalman filters can be utilized to predict future poses for a moving robot by continuously combining measurements from various sources. A Kalman filter is therefore a sensor fusion algorithm. Common sensors are
 
\begin{itemize}
\item motion sensors that estimate position change over time (odometry),
\item laser scanners that scan the environment,
\item cameras that track predefined markers. 
\end{itemize}

By combining the last estimation and the current sensor outputs a new estimation for the current state is obtained. In this process two steps are commonly distinguished: \textit{predict (a priori)} and \textit{update (a posteriori)}.
\\\\
During \textbf{prediction} the next state as well as its covariance are estimated. After that new sensor data is compared with the predicted state, which results in the calculation of the so-called  Gain. The Kalman Gain is an indicator for the certainty of both measurements and prediction. High covariances in the prediction increase the Gain while high covariances in the measurements decrease it.

$$\text{a priori state estimate:} $$
$$\text{a priori covariance estimate:} $$

This will result in the filter trusting either the measurements or the predictions more depending on the gain. 
\\\\
Based on the Gain the Kalman filter produces a posteriori state and covariance estimates during \textbf{update phase}. The higher the gain, the closer these estimates will be to the measurement output and vice versa.

$$\text{a posteriori state estimate:} $$
$$\text{a posteriori covariance estimate:} $$