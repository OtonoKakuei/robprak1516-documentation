\chapter{Evaluation}
\section{Integration}
Due to the consequent usage of the IDSGit the development process can be reviewed in the gitorious commit path. The pure code combiniation of the groups' development branches was quite simple and required only the installation of the dependencies of the groups. After everything compiled fine the real integration work started. All subscribed and published topics had to been updated to match the requirements that had been discussed in advance. The main source of missing specification have been the Frames each Group publishes their data in. Also some Topic names had to be updated since every group used their own topic namespace.
The main integration work done till the end of the project has been to deliver the data the other groups acutally expect to make the Software function as expected.
\section{Test Scenario}
\chapter{Conclusion}
\section{Summary and Conclusion}
\section{Outlook}
In the future the next obvious step is to explore the possibility and requirements for camera tracking of natural markers instead of AR tags. Although AR tags are easy to use and quickly setup, they cannot be attached to arbitrary surfaces, either because of practical or simply because of optical reasons.

It might also be interesting to implement the ability for the robot to resume its previous task, i.e. to find its way back or to directly head to its destination to continue where it left off.